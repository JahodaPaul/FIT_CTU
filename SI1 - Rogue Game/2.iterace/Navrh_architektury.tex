%% ZAKLADNI VLASTNOSTI STRANKY -- VELIKOST PISMA, STRANKY, OKRAJU
\documentclass[12pt,a4paper]{article}
\usepackage[marginparsep=14pt,left=2.5cm,right=2.5cm,top=3.2cm,bottom=4.5cm]{geometry}

\setlength{\parskip}{8pt}
\setlength{\parindent}{25pt}

%% CESTINA
\usepackage[czech]{babel}

%% FONTY
\usepackage[utf8]{inputenc}
\usepackage[T1]{fontenc}

%% VKLADANI ZDROJOVEHO KODU
%\usepackage{listings}

%% ZAHLAVI A ZAPATI
\usepackage{fancyhdr}
\pagestyle{fancy}
\renewcommand{\sectionmark}[1]{\markright{#1}}

% prostredni cast zapati
\cfoot{\thepage}

% leva cast zahlavi -- nazev sekce/subsekce
\lhead{\fancyplain{}{\rightmark}}

\def\picturesfolder{obrazky}

% prava cast zahlavi -- logo fitu
\rhead{\includegraphics[width=5cm]{\picturesfolder/logo}}

%% TABULKY
\usepackage{tabularx}

% podbarveni radku tabulek
\usepackage[table]{xcolor}

%% KRESLENI DIAGRAMU
\usepackage{tikz}

\usepackage{floatrow}
\floatsetup[figure]{capposition=top}

%% PROKLIKAVATELNE ODKAZY
\usepackage[pdftex,pdfpagelabels,bookmarks,hyperindex,hyperfigures]{hyperref}

\hypersetup{
    colorlinks,
    citecolor=blue,
    filecolor=blue,
    linkcolor=blue,
    urlcolor=blue
}


\begin{document}
%%%%% TITLEPAGE

% alternuj bilou a svetle sedou pro radky vsech tabulek
\rowcolors{1}{gray!25}{white}

% bez cislovani stranek
\pagenumbering{gobble}

% bez cary oddelujici zahlavi a zapati
\renewcommand{\headrulewidth}{0pt}
\renewcommand{\footrulewidth}{0pt}

\begin{titlepage}
  % pro zobrazeni loga v zahlavi
  \thispagestyle{fancy}

  % vertikalni zarovnani
	\vspace*{\fill}
	\begin{center}
    {\fontsize{28.83}{100}\selectfont Rogue-like hra}\\[0.6cm]
		{\fontsize{15.74}{40}\selectfont Návrh architektury}\\[1.5cm]
    {\fontsize{10}{10} \selectfont Dokument vytvořen pro potřeby předmětů
    BI-SI1 a BI-SP1}\\
	\end{center}

  % vertikalni zarovnani
	\vspace*{\fill}

  % seznam clenu tymu razeny abecedne podle krestniho jmena
  {\fontsize{10}{10} \selectfont \noindent
\textbf{Autoři:}\\
  Jakub Drbohlav\\
  Jiří Kasl\\
  Pavel Jahoda\\
  Petr Pondělík\\
  Vanda Hendrychová\\
  Vojtěch Pejša\\
  }
\end{titlepage}

\newpage

% cara nahore a dole oddelujici zahlavi a zapati od obsahu stranky
\renewcommand{\headrulewidth}{0.4pt}
\renewcommand{\footrulewidth}{0.4pt}




%%%%% OBSAH

% rimska cisla pro cislovani stranek v obsahu
\pagenumbering{roman}

% samotne vlozeni obsahu
\tableofcontents

\newpage

% zapnout bezne cislovani stranek pomoci arabskych cislic
\pagenumbering{arabic}



%%%%% TEXT

\section{Návrh Architektury}
Architektura aplikace bude fungovat jako MVC(Model-View-Controler). Vybrali jsme
tuto architekturu protože rozděluje aplikaci do tří komponent.

\begin{itemize}
\item Model (správa dat, fyzikální engine)
\item View (vykreslení herních objektů, uživatelské rozhraní)
\item Controller (řídící logika, v našem případě je kontroler velice tenký a v zásadě jen spojuje View a model)
\end{itemize}

Hlavní výhody, které do našeho projektu architektura MVC přináší je snadná práce
v týmu. Programátoři kteří pracují na prezenční vrstvě nemusejí nijak znát
logiku hry a strukturu dat, stačí jim pouze znát navržené rozhraní a
implementovat vůči němu.  Dále nám také architektura umožňuje jednoduše vyměnit
prezenční vrstvu (například přejít na jinou grafickou knihovnu).

Pro lepší představu je architektura MVC znázorněna na následujícím obrázku.
\begin{center}
\includegraphics[width=\textwidth]{\picturesfolder/navrh_architektury}
\end{center}

\subsection{Prezentační vrstva - View}
View je komponenta, která dostává data z modelu a zobrazuje je uživateli. Dále
také zpracovává vstupy od uživatele a na jejich základě volá metody Controlleru.
View je dále rozdělen do \uv{Scén}, příkladem takové scény je například herní
menu, nebo vlastní herní obrazovka.

\subsection{Datová vrstva - Model}
Model reprezentuje veškerá data herního světa a mimo to se i stará a fyzikální
engine hry (například kolize). Model je v našem návrhu plně nezávislý na
zbylých dvou komponentách a komunikace s ním probíhá na základě předem
definovaného rozhraní. Model bude také umožňovat ukládání respektive načítání
hry.

\subsection{Controller}
Controller primárně zajišťuje komunikaci mezi prezenční a datovou vrstvou, v
našem návrhu je velice \uv{tenký}.

\section{Volba knihoven}
Pro implementaci aplikace jsme se rozhodli použít tyto technologie:

\begin{itemize}
\item Správa buildů $\rightarrow$ CMake
\item Grafická knihovna $\rightarrow$ sfml
\item Grafické uživatelské rozhraní $\rightarrow$ dear imgui
\item Fyzika a kolize $\rightarrow$ Box2D
\end{itemize}

Kombinace těchto knihoven můžu v budoucnu umožnit jednoduchý přenos aplikace na
jiný operační systém (zatím je podporován jen Linux), protože všechny námi
zvolené knihovny jsou plně multiplatformní (linux|MacOS|Windows|Android...)
Samozřejmě by například pro implementaci na android bylo potřeba změnit
uživatelské vstupy, to by ale díky plně izolované prezenční vrstvě neměl být
velký problém.

\subsection{build systém}
Jako software na správu buildů jsme zvolili CMake. S nástrojem CMake již byly v
našem týmu částěčné zkušenosti, plně splňuje všechny naše požadavky na build
systém a v současnosti je to pro větší projekty v C++ v zásadě standard.

\subsection{grafická knihovna}
Jako grafické knihovny jsme zvažovali SDL2 a SFML. Obě knihovny nabízejí velice
podobné možnosti, ale s přihlédnutím k jednoduššímu používání a lepší
dokumentaci jsme nakonec zvolili SFML.

\subsection{grafické uživatelské rozhraní}
Vzhledem k tomu, že pro naši hru není uživatelské rozhraní příliš velká
priorita, jsme volili GUI s větším důrazem na jednoduchost použití. \emph{Dear
imgui} je takzvaná immediate gui library, to znamená, že jednotlivé prvky
rozhraní jsou v zásadě bezestavové a je nutné je neustále znovu updatovat a
vykreslovat (i přesto je knihovna velice efektivní z pohledu  využití
procesoru). Tato její vlastnost pro nás ovšem není žádná překážka, protože naše
hra je real time, tudíž zde bude probíhat mnoho updatů za sekundu.  

Naopak přínosem této knihovny je její extrémně jednoduché používání a i zapojení
do projektu. Celá knihovna se v zásadě skládá ze dvou souborů a bindingu, aby
ji bylo možné použít společně se SFML.

\subsection{Fyzikální engine}
Hra má být 2D a to bylo také jedno z klíčových kritérií při výběru fyzikálního
engine. V současnosti sice takových knihoven pro C++ existuje mnoho, ale Box2D
je pravděpodobně nejlepší z nich a má velice kvalitní dokumentaci a tutoriály,
vybrali jsme tudíž tu.

\end{document}
