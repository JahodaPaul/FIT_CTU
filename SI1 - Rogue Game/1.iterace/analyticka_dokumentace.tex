%% ZAKLADNI VLASTNOSTI STRANKY -- VELIKOST PISMA, STRANKY, OKRAJU
\documentclass[12pt,a4paper]{article}
\usepackage[marginparsep=14pt,left=2.5cm,right=2.5cm,top=3.2cm,bottom=4.5cm]{geometry}

\setlength{\parskip}{8pt}
\setlength{\parindent}{25pt}

%% CESTINA
\usepackage[czech]{babel}

%% FONTY
\usepackage[utf8]{inputenc}
\usepackage[T1]{fontenc}

%% VKLADANI ZDROJOVEHO KODU
%\usepackage{listings}

%% ZAHLAVI A ZAPATI
\usepackage{fancyhdr}
\pagestyle{fancy}
\renewcommand{\sectionmark}[1]{\markright{#1}}

% prostredni cast zapati
\cfoot{\thepage}

% leva cast zahlavi -- nazev sekce/subsekce
\lhead{\fancyplain{}{\rightmark}}

% prava cast zahlavi -- logo fitu
\rhead{\includegraphics[width=5cm]{logo}}

\def\picturesfolder{obrazky}

%% TABULKY
\usepackage{tabularx}

% podbarveni radku tabulek
\usepackage[table]{xcolor}

%% KRESLENI DIAGRAMU
\usepackage{tikz}

\usepackage{floatrow}
\floatsetup[figure]{capposition=top}

%% PROKLIKAVATELNE ODKAZY
\usepackage[pdftex,pdfpagelabels,bookmarks,hyperindex,hyperfigures]{hyperref}

\hypersetup{
    colorlinks,
    citecolor=blue,
    filecolor=blue,
    linkcolor=blue,
    urlcolor=blue
}


\begin{document}
%%%%% TITLEPAGE

% alternuj bilou a svetle sedou pro radky vsech tabulek
\rowcolors{1}{gray!25}{white}

% bez cislovani stranek
\pagenumbering{gobble}

% bez cary oddelujici zahlavi a zapati
\renewcommand{\headrulewidth}{0pt}
\renewcommand{\footrulewidth}{0pt}

\begin{titlepage}
  % pro zobrazeni loga v zahlavi
  \thispagestyle{fancy}

  % vertikalni zarovnani
	\vspace*{\fill}
	\begin{center}
    {\fontsize{28.83}{100}\selectfont Rogue-like hra}\\[0.6cm]
		{\fontsize{15.74}{40}\selectfont Analytická dokumentace}\\[1.5cm]
    {\fontsize{10}{10} \selectfont Dokument vytvořen pro potřeby předmětů
    BI-SI1 a BI-SP1}\\
	\end{center}

  % vertikalni zarovnani
	\vspace*{\fill}

  % seznam clenu tymu razeny abecedne podle krestniho jmena
  {\fontsize{10}{10} \selectfont \noindent
\textbf{Autoři:}\\
  Jakub Drbohlav\\
  Jiří Kasl\\
  Pavel Jahoda\\
  Petr Pondělík\\
  Vanda Hendrychová\\
  Vojtěch Pejša\\
  }
\end{titlepage}

\newpage

% cara nahore a dole oddelujici zahlavi a zapati od obsahu stranky
\renewcommand{\headrulewidth}{0.4pt}
\renewcommand{\footrulewidth}{0.4pt}




%%%%% OBSAH

% rimska cisla pro cislovani stranek v obsahu
\pagenumbering{roman}

% samotne vlozeni obsahu
\tableofcontents

\newpage

% seznam obrazku
\listoffigures

\newpage

% zapnout bezne cislovani stranek pomoci arabskych cislic
\pagenumbering{arabic}



%%%%% TEXT

\section{Základní popis hry}
Hra bude určená pro PC, od toho se odvíjí i její ovládání -- primárně klávesnice
s~občasným využitím myši (přesun předmětů v~inventáři). Pohled na herní svět a
postavu bude shora a grafika bude 2D.

\subsection{Herní svět}
Herní svět se bude skládat z~jednotlivých podlaží. Podlaží jsou tvořena
místnostmi, které se vygenerují když hráč poprvé vstoupí na dané podlaží. Při
této příležitosti se také vygenerují NPC a monstra. Obtížnost monster bude
určena na základě úrovně podlaží a úrovně hráče v~okamžiku, kdy poprvé vstoupí
na podlaží. Přechod mezi podlažími bude umožněn pomocí schodů.

Mezi místnostmi ve stejném podlaží se dá pohybovat pomocí dveří. Všechny
místnosti mají stejnou velikost a tvar, liší se však vzhledem, nepřáteli a
překážkami uvnitř místnosti. Ve většině místností se budou nacházet nepřátelé.
Některé místnosti mohou být prázdné a v~části budou spřátelené postavy (například
obchodník), se kterými bude moci hráč obchodovat nebo od nich dostat zadání
questu (úkolu).

Tento svět bude pseudonáhodně generovaný počítačem na začátku hry, a tudíž se
může pro jednotlivé hry lišit.

\subsection{Postava a systém boje}
Postava, za kterou bude hráč hrát je trpaslík. Tento trpaslík bude moci bojovat
pouze na dálku vystřelováním projektilů, které po první kolizi zmizí. V~případě
kolize s~nepřítelem mohou nepřítele zranit. Možné zbraně budou magie, luky a
kuše a vrhací zbraně (vrhací nože, granáty).

\subsection{Cíl hry}
Hra začíná v~trpasličím městě v~horních částech podzemí hory (podlaží 0).
Celá tato oblast je bez nepřátel a hráč se zde poprvé seznámí s~herními
mechanismy.  Mimo  jiné ovšem také bude sloužit jako herní menu.
Zde se hráč poprvé seznámí s~příběhem. Pomocí dialogu s~NPC se dozví,
že hluboko v~hoře pod městem se usídlil drak, který vesnici ohrožuje.

Také zde bude možnost změnit některá nastavení (reprezentováno jako dialogy
s~NPC). Jedním z~nich bude i cíl hry, který v~našem případě reprezentuje
herní obtížnost. Ve všech případech bude potřeba odstranit nebezpečí,
které vesnici ohrožuje, tedy draka.  Možností jak na to ale bude víc:

\begin{itemize}
\item Například drakovi podat silný lektvar spánku (lehká) a tím ho uspat na
příštích deset let.
\item Draka draka vyhnat pryč z~hory (střední).
\item Draka zabít a tím odstranit nebezpečí jednou provždy (těžká).
\end{itemize}

Toto bude skutečně nastavení obtížnosti, tudíž ovlivní například zisk zlata,
nebo sílu generovaných nepřátel, ale zároveň také samotný cíl hry
a herní skóre.

\newpage
\section{Procesy ve hře}
Kapitola popisuje procesy ve hře.
\begin{itemize}
  \item{Část \ref{procesy:obchod} nazvaná \uv{Obchodování} se týká procesů,
    které se stávají při prodeji a nákupu produktů.}
  \item{Část \ref{procesy:questy} nazvaná \uv{Questy} popisuje procesy, které
    nastávají od inicializace questu, až po jeho dokončení.}
\end{itemize}

\subsection{Obchodování} \label{procesy:obchod}
Hra bude obsahovat obchodní svět, který umožní hráči nakupovat a prodávat
produkty popřípadě služby. Na rozdíl od ostatních her, které v~naprosté většině
případů obsahují NPC, které s~radostí odkoupí jakýkoliv předmět vlastněný hráčem
se naše hra pokusí o~více realistický model, který umožní NPC odmítnutí
nabízeného zboží.

Materiály, které má NPC u~sebe budou ovlivňovat i to, jaké předměty bude
obchodník vyrábět a tudíž prodávat.

\begin{figure}
\begin{center}
\includegraphics[width=\textwidth]{\picturesfolder/Proces_NPC}
  \caption{Model procesu obchodování}
  \label{process:trade}
\end{center}
\end{figure}

\subsection{Questy} \label{procesy:questy}
Quest je hráči přidělen tak, že hráč přijde k~NPC postavě a potvrdí zadání
questu. Dále hráč plní úkony potřebné k~vyhotovení questu. Tyto úkony mají
několik typů, například zabíjení monster nebo sběr předmětu. Po vyhotovení všech
úkonů potřebných pro splnění questu si může hráč vyzvednout odměnu od NPC,
u~kterého quest přijal.

\begin{figure}
\begin{center}
\includegraphics[width=\textwidth]{\picturesfolder/Procesy_Questy}
  \caption{Model procesu questu}
  \label{process:quest}
\end{center}
\end{figure}

\newpage


\section{Doménový model ve hře}
Kapitola popisuje třídy (entity), které souvisejí s~analyzovanou doménou.
Jednotlivé třídy jsou zde detailně popsány tak, aby bylo zřejmé, jaké všechny
objekty a informace budou ve hře používány.

Entity jsou rozděleny do dvou částí na základě své role v~herním světě. V~části
Postavy a předměty je obsažen popis entit souvisejících s~postavami a předměty
ve hře.  V~části Herní svět je pak obsažen popis entit herního světa.

Propojení těchto dvou částí je realizováno přes předměty obsažené v~entitě
RG:Kontejner a Entity (jakožto živé herní objekty -- RG:Entita) obsažené
v~RG:Místnost.

\bigskip
\noindent
\subsection*{Omezení na doménový model:}

Doménový model bez dodaného omezení umožňuje, aby existovala instance objektu,
který není přítomný v~kontejneru, u~hráče ani u~obchodníka. Zároveň umožňuje,
aby tento předmět byl přítomen u~více z~těchto prvků najednou. Proto je dodáno
omezení, že předmět se může nacházet právě u~jednoho z~těchto prvků.

\subsection{Postavy a předměty}
Tato kapitola obsahuje popis entit reprezentujících prvky herního světa, které
budou přímo tvořit nestatický a hratelný obsah herního světa (tedy vždy budou
disponovat vlastnostmi, které mohou ovlivnit ostatní entity či s~nimi bude možné
provést interakci, která bude mít vliv na hráče -- např. NPC a úkoly, obchod).
Diagram zobrazující doménový model postav a předmětů se nachází na obrázku
\ref{domenovy:postavy}.

\subsubsection{Entita}
Jedná se o~živou bytost. V~aplikaci je reprezentována především NPC, hráčem či
nepřítelem.

\def\arraystretch{1.75}
\noindent
\begin{tabularx}{\textwidth}{|l|>{\raggedright}X|}
\hline
\cellcolor{black!80}\textcolor{gray!25}{\textbf{Atribut}} & \cellcolor{black!80}\textcolor{gray!20}{\textbf{Popis}}\tabularnewline \hline
 HP & Životy entity. \tabularnewline
\hline
 Jméno & Jméno entity. \tabularnewline
\hline
 Level & Level entity. \tabularnewline
\hline
 Obrana & Obrana entity. Čím vyšší má entita obranu,
                tím méně ji zraňují nepřátelské útoky. \tabularnewline
\hline
 Pozice & Pozice entity v~místnosti. \tabularnewline
\hline
 Rychlost & Rychlost pohybu entity. \tabularnewline
\hline
 Útok & Útok entity. Udává, kolik entita ubírá životů ostatním entitám.
\tabularnewline\hline
\end{tabularx}

\begin{figure}
\begin{center}
  \includegraphics[width=\textwidth]{\picturesfolder/Domenovy_model_2}
  \caption{Doménový model -- Postavy a předměty}
  \label{domenovy:postavy}
\end{center}
\end{figure}

\subsubsection{Nepřítel}
Agresivní stvoření, které bude útočit na hráče.

\noindent
\begin{tabularx}{\textwidth}{|l|>{\raggedright}X|}
\hline
\cellcolor{black!80}\textcolor{gray!25}{\textbf{Atribut}} & \cellcolor{black!80}\textcolor{gray!20}{\textbf{Popis}}\tabularnewline \hline
 Předmět drop & Určuje, jaká je šance, že po zabití nepřítele
  z~něho  vypadne předmět. \tabularnewline
\hline
 XP drop & Udává, kolik zkušeností dostane hráč za zabití
  nepřítele.
\tabularnewline\hline
\end{tabularx}

\subsubsection{Hráč}
Hlavní hrdina celé hry. Postavička trpaslíka, kterou bude ovládat hráč.

\noindent
\begin{tabularx}{\textwidth}{|l|>{\raggedright}X|}
\hline
\cellcolor{black!80}\textcolor{gray!25}{\textbf{Atribut}} & \cellcolor{black!80}\textcolor{gray!20}{\textbf{Popis}}\tabularnewline \hline
 Peníze & Udává, kolik peněz má hráč u~sebe. Za peníze si hráč může
             kupovat u~obchodníku předměty. \tabularnewline
\hline
 XP & Udává, kolik zkušeností má hráč. Podle toho se určuje jeho
  level.
\tabularnewline\hline
\end{tabularx}

\subsubsection{NPC}
Postava, která bude ve hře, aby nasměrovala nebo zadala úkol hráči.

\noindent
\begin{tabularx}{\textwidth}{|l|>{\raggedright}X|}
\hline
\cellcolor{black!80}\textcolor{gray!25}{\textbf{Atribut}} & \cellcolor{black!80}\textcolor{gray!20}{\textbf{Popis}}\tabularnewline \hline
 Dialog & Text, který se vypíše při rozhovoru s~hráčem. Při této interakci
může dojít k~nápovědě hráče nebo zadání ukolu.
\tabularnewline\hline
\end{tabularx}

\subsubsection{Obchodník}
Specializace NPC. Hráč od něj bude moci kupovat nebo mu prodávat předměty.
Obchodník zprostředkovává kontakt hráče s~obchodním systémem.

\subsubsection{Úkol/Quest}
Výzva, která bude zadaná některou z~NPC postav.

\noindent
\begin{tabularx}{\textwidth}{|l|>{\raggedright}X|}
\hline
\cellcolor{black!80}\textcolor{gray!25}{\textbf{Atribut}} & \cellcolor{black!80}\textcolor{gray!20}{\textbf{Popis}}\tabularnewline \hline
 Stav & Udává, v~jakém stavu se nachází úkol.
\tabularnewline\hline
\end{tabularx}

\bigskip
\noindent
Stavy, ve kterých se úkol může nacházet, jsou zachyceny na obrázku
\ref{stavy:ukol}.

\begin {itemize}
\item{\textbf{Nepřijatý:} Hráč vede dialog s~NPC, který zobrazí možnost dialogu
  samotného úkolu. Tedy hráč nyní ví o~existence úkolu, ale úkol nepřijal.}
\item{\textbf{Přijatý:} Hráč v~dialogu přijal úkol. Úkol může být přerušen a tím
  se vrátit zpět do stavu nepřijatý.}
\item{\textbf{Splněný:} Hráč úspěšně dokončil úkol (např. splnil požadavky pro
  jeho dokončení a pomocí dialogu to oznámil NPC)}
\item{\textbf{Nesplněný:} Hráč neúspěšně dokončil úkol (např. vypršel čas). Tato
  možnost nebude přítomna u~všech úkolů.}
\item{\textbf{Odmítnutý:} Hráč v~dialogu výslovně odmítne vykonávat úkol. Tím už
  úkol není možné přijmout. Toto může mít dopad např. na hráčovu reputaci
v~podlaží.}
\end {itemize}

\begin{figure}
\begin{center}
  \includegraphics[width=\textwidth]{\picturesfolder/State_diagram_ukol}
  \caption{Diagram stavů -- úkol}
  \label{stavy:ukol}
\end{center}
\end{figure}

\subsubsection{Efekt}
Entitě, na kterou bude působit, změní vlastnosti a schopnosti.

\noindent
\begin{tabularx}{\textwidth}{|l|>{\raggedright}X|}
\hline
\cellcolor{black!80}\textcolor{gray!25}{\textbf{Atribut}} & \cellcolor{black!80}\textcolor{gray!20}{\textbf{Popis}}\tabularnewline \hline
 Síla & Udává, jak velký vliv má efekt na entitu.\tabularnewline
\hline
 Trvání & Udává, jak dlouho daný efekt trvá.\tabularnewline
\hline
 Účinek & Udává, jaký účinek má efekt. Může jít například o~slepotu, otravu,
           zesílení, zrychlení,$\ldots$
\tabularnewline\hline
\end{tabularx}

\subsubsection{Předmět}
Věc, kterou bude hráč moci použít ke zlepšení svých schopností nebo zničení
nepřátel.

\noindent
\begin{tabularx}{\textwidth}{|l|>{\raggedright}X|}
\hline
\cellcolor{black!80}\textcolor{gray!25}{\textbf{Atribut}} & \cellcolor{black!80}\textcolor{gray!20}{\textbf{Popis}}\tabularnewline \hline
 Síla & Udává, jak velký vliv má efekt na hráčovy schopnosti či na
  nepřítele. \tabularnewline
\hline
 Trvání & Udává, jak dlouho daný efekt trvá. \tabularnewline
\hline
 Účinek & Udává, jaký účinek má efekt. Například slepotu, otravu, zesílení,
             zrychlení atd$\ldots$
\tabularnewline\hline
\end{tabularx}

\subsubsection{Spotřebitelný}
Specializace předmětu, který bude mít omezený počet použití, než bude vyplýtván
a zmizí hráči z~inventáře.

\noindent
\begin{tabularx}{\textwidth}{|l|>{\raggedright}X|}
\hline
\cellcolor{black!80}\textcolor{gray!25}{\textbf{Atribut}} & \cellcolor{black!80}\textcolor{gray!20}{\textbf{Popis}}\tabularnewline \hline
 Počet použití & Udává, kolikrát se dá daný předmět používat, než zmizí
                   z~inventáře.\tabularnewline
\hline
 Účinek & Udává, jaký účinek má předmět. Například léčení, zranění,
             rozšíření statistik atp.
\tabularnewline\hline
\end{tabularx}

\subsubsection{Vybavení}
Specializace předmětu, kterou si bude moci hráč na sebe nasadit a která mu bude
zvyšovat schopnosti do té doby, dokud jí bude mít vybavenou.

\noindent
\begin{tabularx}{\textwidth}{|l|>{\raggedright}X|}
\hline
\cellcolor{black!80}\textcolor{gray!25}{\textbf{Atribut}} & \cellcolor{black!80}\textcolor{gray!20}{\textbf{Popis}}\tabularnewline \hline
 Statistiky & Udává, které staty se o~kolik změní při nasazení tohoto
  předmětu.
\tabularnewline\hline
\end{tabularx}

Každé vybavení přítomné ve hře se může nacházet v~různých stavech, které jsou
zachyceny na obrázku \ref{stavy:vybaveni}.

\begin {itemize}
\item{\textbf{Použitelné:} Hráč získal vybavení a toto vybavení nebylo poškozeno
  do té míry, kdy není použitelné.}
\item{\textbf{Poškozené:} Vybavení bylo poškozeno za hranici použitelnosti. Lze
  znovu opravit a získat tak použitelné vybavení.}
\item{\textbf{Zahozené:} Vybavení bylo zahozeno z~inventáře. Tím zmizí z~herního
  světa.}
\end {itemize}

\begin{figure}
\begin{center}
  \includegraphics[width=\textwidth]{\picturesfolder/State_diagram_vybaveni}
  \caption{Diagram stavů -- vybavení}
  \label{stavy:vybaveni}
\end{center}
\end{figure}

\subsection{Herní svět}
Tato kapitola obsahuje popis entit herního světa. Jedná se o~entity, které budou
tvořit strukturu a výplň samotného herního světa. Některé z~těchto objektů budou
umožňovat interakci.  Diagram zobrazující doménový model herního světa se
nachází na obrázku \ref{domenovy:hra}.

\begin{figure}
\begin{center}
  \includegraphics[width=\textwidth]{\picturesfolder/Domenovy_model}
  \caption{Doménový model -- Herní svět}
  \label{domenovy:hra}
\end{center}
\end{figure}

\subsubsection{Herní svět}
Herní svět se skládá z~několika podlaží.

\noindent
\begin{tabularx}{\textwidth}{|l|>{\raggedright}X|}
\hline
\cellcolor{black!80}\textcolor{gray!25}{\textbf{Atribut}} & \cellcolor{black!80}\textcolor{gray!20}{\textbf{Popis}}\tabularnewline \hline
 Název & Název herního světa.
\tabularnewline\hline
\end{tabularx}

\subsubsection{Podlaží}
Každé podlaží se skládá z~rozsáhlého labyrintu místností.

\noindent
\begin{tabularx}{\textwidth}{|l|>{\raggedright}X|}
\hline
\cellcolor{black!80}\textcolor{gray!25}{\textbf{Atribut}} & \cellcolor{black!80}\textcolor{gray!20}{\textbf{Popis}}\tabularnewline \hline
 Název & Název podlaží.
\tabularnewline\hline
\end{tabularx}

\subsubsection{Místnost}
V~každé místnosti bude hráč čelit různým výzvám, od nepřátel až po
překážky.

\noindent
\begin{tabularx}{\textwidth}{|l|>{\raggedright}X|}
\hline
\cellcolor{black!80}\textcolor{gray!25}{\textbf{Atribut}} & \cellcolor{black!80}\textcolor{gray!20}{\textbf{Popis}}\tabularnewline \hline
 Název & Název místnosti \tabularnewline
\hline
 Entity & Místnosti obsahuje entity. \tabularnewline
\hline
 Pozadí & Místnost má určité pozadí místnosti.
\tabularnewline\hline
\end{tabularx}

\subsubsection{Objekt}
Objekt bude umístěný v~místnosti.

\noindent
\begin{tabularx}{\textwidth}{|l|>{\raggedright}X|}
\hline
\cellcolor{black!80}\textcolor{gray!25}{\textbf{Atribut}} & \cellcolor{black!80}\textcolor{gray!20}{\textbf{Popis}}\tabularnewline \hline
 Pozice & Obsahuje pozici objektu v~místnosti.
\tabularnewline\hline
\end{tabularx}

\subsubsection{Dveře}
Specializace objektu. Díky dveřím se bude moci hráč pohybovat mezi jednotlivými
místnostmi a podlažími.

\noindent
\begin{tabularx}{\textwidth}{|l|>{\raggedright}X|}
\hline
\cellcolor{black!80}\textcolor{gray!25}{\textbf{Atribut}} & \cellcolor{black!80}\textcolor{gray!20}{\textbf{Popis}}\tabularnewline \hline
 Cílová místnost & Obsahuje místnost, do které se dá dveřmi dostat.
\tabularnewline\hline
\end{tabularx}

\subsubsection{Past}
Specializace objektu, která způsobí hráči zranění, když se dostane na stejnou
pozici jako má past.

\noindent
\begin{tabularx}{\textwidth}{|l|>{\raggedright}X|}
\hline
\cellcolor{black!80}\textcolor{gray!25}{\textbf{Atribut}} & \cellcolor{black!80}\textcolor{gray!20}{\textbf{Popis}}\tabularnewline \hline
 Zranění & Udává, jaké zranění způsobí past hráči, jestliže na ní
  vkročí.
\tabularnewline\hline
\end{tabularx}

\subsubsection{Stavba}
Specializace objektu, jedná o~stavby, se kterými se hráč může dostat do
interakce. To může způsobit získání nějakého pozitivní nebo negativního efektu,
nebo objevení se nepřítele. Jedná se například o~kapličku, u~které se může hráč
pomodlit.

\noindent
\begin{tabularx}{\textwidth}{|l|>{\raggedright}X|}
\hline
\cellcolor{black!80}\textcolor{gray!25}{\textbf{Atribut}} & \cellcolor{black!80}\textcolor{gray!20}{\textbf{Popis}}\tabularnewline \hline
 Dialog & Text, který se vypíše hráči při interakci se stavbou.
\tabularnewline\hline
\end{tabularx}

\subsubsection{Kontejner}
Specializace předmětu, ve které se mohou nacházet různé předměty.

\noindent
\begin{tabularx}{\textwidth}{|l|>{\raggedright}X|}
\hline
\cellcolor{black!80}\textcolor{gray!25}{\textbf{Atribut}} & \cellcolor{black!80}\textcolor{gray!20}{\textbf{Popis}}\tabularnewline \hline
 Předmět & Předmět, který z~kontejneru vypadne, jestliže ho hráč
  otevře.
\tabularnewline\hline
\end{tabularx}

\newpage

\section{Model požadavků}
Tato kapitola obsahuje požadavky kladené na nově vznikající hru. Požadavky jsou
rozděleny na funkční a nefunkční.

\subsection{Funkční požadavky}

\subsubsection{Ovládání běhu hry}
Hráč bude moci ovládat běh hry. Tento aspekt hry bude zpřístupněný pomocí
funkcionalit, které umožní začít novou hru, ukončit hru, uložit a načíst hru a
hru pozastavit.

\subsubsection{Ovládání herní postavy a interakce s~herním světem}
Hráč bude moci ovládat herní postavu. Ovládání herní postavy úzce souvisí
s~požadavkem na interakci s~herním světem, jelikož akce herní postavy mohou vyvolat
interakci s~tímto světem. Tato rozsáhlejší funkce hry bude umožněna pohybem
(pohyb herní postavy napříč herní místností a přesun herní postavy mezi
jednotlivými místnostmi).

Dále bude hráč mít možnost útočit na nepřátele.

Herní postava bude mít možnost sbírat předmět (předměty se mohou objevit po
zabití nepřítele, nebo ležet v~místnosti hry již od jejího vygenerování). Tyto
předměty půjdou prodat NPC obchodníkovi za využití herního obchodu.

\subsubsection{Správa inventáře herní postavy}
Hráč bude moci otevřít/uzavřít inventář postavy. Uvnitř inventáře bude mít
zobrazené všechny předměty, které aktuálně vlastní. Všechny předměty bude moct
hráč z~inventáře zahodit na zem. Vybrané předmětu půjdou aktivně využít (např.:
lektvar léčení), nebo obléknout (nová zbroj).

\subsubsection{Vylepšení herní postavy}
Hra bude hráči umožňovat vylepšit jeho postavu. Toto bude umožněno
prodejem/výměnou herních surovin s~NPC obchodníkem a následným nakoupením lepší
výbavy.  Herní suroviny se budou nacházet volně na mapě, také je bude možné
získat poražením nepřátel.

\subsubsection{Nastavení hry}
Nastavení hry bude rozděleno do dvou kategorií.

\begin{enumerate}
  \item{\textbf{Nastavení při zahájení nové hry} -- Při startu nové hry si hráč
    zvolí jméno postavy, její vzhled a obtížnost hry.}
  \item{\textbf{Globální nastavení} -- V~hlavním menu bude možné kdykoli
v~průběhu hraní nastavit parametry hry. Například nastavit rozlišení,
    hlasitost, nebo zapnout/vypnout hudbu.}
\end{enumerate}

\subsubsection{Přístup do informační sekce hry}
Hra umožní zobrazovat statistiky o~postavě (množství poražených nepřátel,
celkem vyléčených životů$\dots$) Dále si hráč během hry bude moci zobrazit statickou
herní nápovědu. Některé předměty také budou mít interaktivní nápovědu, která se
zobrazí po najetí myši na předmět v~inventáři.

Další informační prvek bude minimapa. Minimapa bude obsahovat všechy části herní
mapy, které již hráč navštívil.

\begin{figure}
\begin{center}
  \includegraphics[width=0.75\textwidth]{\picturesfolder/uc_tabulka}
  \caption{Tabulka ukazující vztah mezi požadavky a případy použití.}
  \label{uc:tabulka}
  \end{center}
\end{figure}

\subsection{Nefunkční požadavky}

\subsubsection{N1 --  Grafické uživatelské rozhraní  }
  Hra bude mít 2D grafiku s~pohledem shora. Ovládání bude bude umožněno
  primárně s~pomocí klávesnice.

\subsubsection{N2 -- Žánr hry}
  Hra bude real-time RPG s~důrazem na obchod a směnu zboží a předmětů s~herními
  postavami.

\subsubsection{N3 -- Cílová platforma}
Hra bude plně funkční na PC pod operačním systémem linux. Instalace bude
umožněna pomocí .deb instalační systému. (debian)

\subsubsection{N4 -- Jazyk hry}
Hra bude v~anglickém jazyce (dialogy, předměty, menu$\dots$).

\newpage

\section{Model případů užití}
V~této kapitole jsou popsány případy užití hry. Jedné se o~funkcionality, které
bude vyvíjená hra poskytovat hráči. Dále tato kapitola obsahuje popis všech
účastníků.

\subsection{Účastníci}
Tato kapitola obsahuje popis účastníků běhu hry.

\subsubsection{Hráč}
Osoba pohybující se v~rozhraní hry a ovládající herní postavu. Hráč využívá
funkcionalit poskytovaných hrou.


\subsection{Případy užití}
Kapitola obsahuje popisuje případy užití hry. Jedná se o~funkcionality související s:

\begin{itemize}
  \item{\textbf{Řízením běhu hry} -- zde lze pro příklad uvést start nebo konec hry, herní
    pauzu atp.}
  \item{\textbf{Manipulací s~herními objekty} -- například pohyb herní postavy, sbírání
    předmětů, užití předmětů, útok na nepřítele atp.}
  \item{\textbf{Prací s~herním rozhraním} -- například zobrazení inventáře,
    zobrazení mapy, vyvolání nápovědy, nastavení atp.}
\end{itemize}

\subsubsection{Řízení běhu hry}

Kapitola obsahuje popis funkčností hry souvisejících s~řízením běhu hry.
Na obrázku \ref{uc:rizenibehu} najdete diagram.

\begin {itemize}
\item{\textbf{Spuštění/ukončení hry} --  V~hlavní nabídce hry bude hráči
  umožněno spustit novou hru, případně aplikaci vypnout. Novou hru bude možné
  zahájit také také přechodem do hlavní nabídky z~právě probíhající hry, stejně
  tak bude v~tomto případě umožněno vypnout aplikaci.}
\item{\textbf{Pozastavení/pokračovaní hry} --  Hráč bude moci hru pozastavit a
  následně se dohry navrátit.}
\item{\textbf{Uložení hry} --  Hráč bude moci uložit herní postup. Tato
  funkcionalita bude založena na principu jediné uložitelné pozice, která bude
v~případě smrti herní postavy vymazána.}
\item{\textbf{Načtení hry} --  Hráč bude moci načíst herní postup. Tato
  funkcionalita bude založena na principu, kdy se uložená hra po načtení vymaže
  (z~důvodu zamezení zkoumání neznámých předmětů stylem uložení/načtení).}
\end {itemize}

\begin{figure}
\begin{center}
  \includegraphics[width=.6\textwidth]{\picturesfolder/Model_pripadu_uziti_I}
  \caption{Model případů užití -- Řízení běhu hry}
  \label{uc:rizenibehu}
\end{center}
\end{figure}

\subsubsection{Manipulace s~herními objekty}

Kapitola obsahuje popis funkčností hry souvisejících s~manipulací s~herními
objekty. Na obrázku \ref{uc:manipulace} najdete diagram.

\begin {itemize}
\item{\textbf{Pohyb} --  Hráč se bude pohybovat po mapě pomocí kláves šipek
  (případně WASD,toto bude umožněno změnit v~nastavení hry). Bude se moci
  pohybovat v~pravoúhlém (tedy doleva, dolů, doprava, nahoru) či diagonálně
  (kombinace dvou pohybových kláves).}
\item{\textbf{Útok} --  Hráč bude moci provést útok ve formě výstřelu, který
v~případě zásahu neživého objektu zanikne a v~případě zásahu nepřítele tohoto
  nepřítele zraní (v~závislosti na použité zbrani, vlastnostech postavy a
  vlastnostech nepřítele).}
\item{\textbf{Sbírání předmětů} --  Hráč bude mít možnost sebrat předmět
  nacházející se v~místnosti. Tento předmět se hráči objeví v~inventáři. Předmět
  se bude nacházet buď volně v~místnosti, či se objeví po zabití nepřítele.}
\item{\textbf{Použití předmětu} --  Hráči bude umožněno použít některé
z~předmětů přítomných v~inventáři.}
\item{\textbf{Zahození předmětu} --  Hráči bude umožněno zahodit předmět
z~inventáře.}
\item{\textbf{Obchodovat s~NPC} --  Hráč bude moci vstoupit do dialogu obchodu
  s~NPC postavou nacházející se v~herní místnosti.}
\end {itemize}

\begin{figure}
\begin{center}
  \includegraphics[width=.6\textwidth]{\picturesfolder/Model_pripadu_uziti_II}
  \caption{Model případů užití -- Manipulace s~herními objekty}
  \label{uc:manipulace}
\end{center}
\end{figure}

\subsubsection{Práce s~herním rozhraním}

Kapitola obsahuje popis funkčností hry souvisejících s~prací s~herním rozhraním.
Na obrázku \ref{uc:rozhrani} najdete diagram.

\begin {itemize}
\item{\textbf{Otevření/zavření inventáře} --  Hráč bude moci otevírat či zavírat
  inventář,přičemž v~inventáři mu bude k~dispozici seznam předmětů, které má
  hráč u~sebe.  Některé předměty bude možné interaktivně využít.}
\item{\textbf{Zobrazení statistik postavy} --  Hráči bude umožněno zobrazit si
  statistiky herní postavy.}
\item{\textbf{Zobrazení mapy} --  Hráči bude umožněno zobrazit mapu herních
  místností s~vyznačením jeho aktuální polohy.}
\item{\textbf{Nastavení hry} --  Hráči bude umožněno změnit nastavení hry. Jedná
  se například o~změnu obtížnosti hry, změnu rozlišení herního okna atp.}
\item{\textbf{Zobrazení statické nápovědy} --  Hráč si bude moci kdykoliv
v~průběhu hry zobrazit nápovědu, ta bude obsahovat informaci vztahující se ke hře
  globálně.  Například ovládání, nebo legendu k~minimapě.}
\item{\textbf{Nejvyšší skóre} --  Hráči bude umožněno zobrazit seznam svých
  nejlepších dosažených výsledků.}
\end {itemize}

\begin{figure}
\begin{center}
  \includegraphics[width=.6\textwidth]{\picturesfolder/Model_pripadu_uziti_III}
  \caption{Model případů užití -- Práce s~herním rozhraním}
  \label{uc:rozhrani}
\end{center}
\end{figure}

\newpage

\section{Návrhy herních obrazovek}
\begin{itemize}
  \item{\textbf{Úvodní obrazovka} -- Obrazovka s~jednoduchou grafikou, která se
    zobrazí po spuštění hry. Obsahuje hlavní menu hry. (Obrázek
    \ref{screen:intro})}
  \item{\textbf{Nová hra} -- Na této obrazovce si hráč zvolí své jméno a
    obtížnost hry; následně také spustí hru.(Obrázek \ref{screen:newgame})}
  \item{\textbf{Nastavení} -- globální nastavení hry (hlasitost,
    ovládání,$\dots$)(Obrázek \ref{screen:settings})}
  \item{\textbf{Inventář} -- zobrazení inventáře v~průběhu hry. Hráč zde vidí
    všechny předměty, které ve hře nasbíral (může je
    aktivovat/deaktivovat)(Obrázek \ref{screen:inventory})}
\end{itemize}

\begin{figure}
\begin{center}
  \includegraphics[width=\textwidth]{\picturesfolder/Design_StartScreen}
  \caption{Návrh úvodní obrazovky}
  \label{screen:intro}
\end{center}
\end{figure}

\begin{figure}
\begin{center}
  \includegraphics[width=\textwidth]{\picturesfolder/Design_NewGameScreen}
  \caption{Návrh obrazovky nové hry}
  \label{screen:newgame}
\end{center}
\end{figure}

\begin{figure}
\begin{center}
  \includegraphics[width=\textwidth]{\picturesfolder/Design_SettingsScreen}
  \caption{Návrh obrazovky nastavení}
  \label{screen:settings}
\end{center}
\end{figure}

\begin{figure}
\begin{center}
  \includegraphics[width=\textwidth]{\picturesfolder/Design_Inventory}
  \caption{Návrh obrazovky inventáře}
  \label{screen:inventory}
\end{center}
\end{figure}

\end{document}
