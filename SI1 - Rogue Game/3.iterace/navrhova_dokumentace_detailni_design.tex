%% ZAKLADNI VLASTNOSTI STRANKY -- VELIKOST PISMA, STRANKY, OKRAJU
\documentclass[12pt,a4paper]{article}
\usepackage[marginparsep=18pt,left=2.5cm,right=2.5cm,top=3.2cm,bottom=4.5cm]{geometry}

\setlength{\parskip}{8pt}
\setlength{\parindent}{25pt}

%% CESTINA
\usepackage[czech]{babel}

%% FONTY
\usepackage[utf8]{inputenc}
\usepackage[T1]{fontenc}

%% VKLADANI ZDROJOVEHO KODU
%\usepackage{listings}

%% ZAHLAVI A ZAPATI
\usepackage{fancyhdr}
\pagestyle{fancy}
\renewcommand{\sectionmark}[1]{\markright{#1}}

% prostredni cast zapati
\cfoot{\thepage}

% leva cast zahlavi -- nazev sekce/subsekce
\lhead{\fancyplain{}{\rightmark}}

\def\picturesfolder{obrazky}

% prava cast zahlavi -- logo fitu
\rhead{\includegraphics[width=5cm]{\picturesfolder/logo}}

%% TABULKY
\usepackage{tabularx}

% podbarveni radku tabulek
\usepackage[table]{xcolor}

%% KRESLENI DIAGRAMU
\usepackage{tikz}

\usepackage{floatrow}
\floatsetup[figure]{capposition=top}

%% PROKLIKAVATELNE ODKAZY
\usepackage[pdftex,pdfpagelabels,bookmarks,hyperindex,hyperfigures]{hyperref}

\hypersetup{
    colorlinks,
    citecolor=blue,
    filecolor=blue,
    linkcolor=blue,
    urlcolor=blue
}


\begin{document}
%%%%% TITLEPAGE

% alternuj bilou a svetle sedou pro radky vsech tabulek
\rowcolors{1}{gray!25}{white}

% bez cislovani stranek
\pagenumbering{gobble}

% bez cary oddelujici zahlavi a zapati
\renewcommand{\headrulewidth}{0pt}
\renewcommand{\footrulewidth}{0pt}

\begin{titlepage}
  % pro zobrazeni loga v zahlavi
  \thispagestyle{fancy}

  % vertikalni zarovnani
	\vspace*{\fill}
	\begin{center}
    {\fontsize{28.83}{100}\selectfont Rogue-like hra}\\[0.6cm]
		{\fontsize{15.74}{40}\selectfont Návrhová dokumentace -- detailní design}\\[1.5cm]
    {\fontsize{10}{10} \selectfont Dokument vytvořen pro potřeby předmětů
    BI-SI1 a BI-SP1}\\
	\end{center}

  % vertikalni zarovnani
	\vspace*{\fill}

  % seznam clenu tymu razeny abecedne podle krestniho jmena
  {\fontsize{10}{10} \selectfont \noindent
\textbf{Autoři:}\\
  Jakub Drbohlav\\
  Jiří Kasl\\
  Pavel Jahoda\\
  Petr Pondělík\\
  Vanda Hendrychová\\
  Vojtěch Pejša\\
  }
\end{titlepage}

\newpage

% cara nahore a dole oddelujici zahlavi a zapati od obsahu stranky
\renewcommand{\headrulewidth}{0.4pt}
\renewcommand{\footrulewidth}{0.4pt}




%%%%% OBSAH

% rimska cisla pro cislovani stranek v obsahu
\pagenumbering{roman}

% samotne vlozeni obsahu
\tableofcontents

\newpage

% zapnout bezne cislovani stranek pomoci arabskych cislic
\pagenumbering{arabic}



%%%%% TEXT

\section{Realizace případů užití (model komunikace)}

Kapitola popisuje komunikaci tříd v rámci vybraných případů užití.  Tato
kapitola zahrnuje pouze popis spolupráce tříd, které jsou důležité z hlediska
objektově orientovaného návrhu.

\subsection{Realizace případu užití: Start hry}

Tato sekce obsahuje popis spolupráce tříd pro realizaci případu užití
\textbf{Start hry}. Následující diagram zachycuje spolupráci tříd při spuštění
hry. Spuštění hry představuje odchycení události stisku tlačítka New game v
nabídce hry a následný přechod do herní scény.

\begin{center}
\includegraphics[width=\textwidth]{\picturesfolder/Flow_diagram01_starthry.pdf}
\end{center}

\subsection{Realizace případu užití: Konec hry}

Tato sekce obsahuje popis spolupráce tříd pro realizaci případu užití
\textbf{Konec hry}. Následující diagram zachycuje spolupráci tříd při ukončení.
Ukončení hry představuje odchycení události stisku tlačítka Exit v nabídce hry a
následné ukončení programu.

\begin{center}
\includegraphics[width=\textwidth]{\picturesfolder/Flow_diagram02_konechry.pdf}
\end{center}

\subsection{Realizace případu užití: Pohyb}

Tato sekce obsahuje popis spolupráce tříd pro realizaci případu užití
\textbf{Pohyb}. Následující diagram zachycuje spolupráci tříd při pohybu. Pohyb
představuje odchycení události stisku tlačítek pro pohyb, uvedení entity hráče
do pohybu, propagaci změn pozice hráče při pohybu do modelu a následné
vykreslení hráče na aktualizované pozici.

\begin{center}
\includegraphics[width=\textwidth]{\picturesfolder/Flow_diagram03_pohyb.pdf}
\end{center}

\subsection{Realizace případu užití: Nastavení}

Tato sekce obsahuje popis spolupráce tříd pro realizaci případu užití
\textbf{Nastavení}.  Následující diagram zachycuje spolupráci tříd při
Nastavení. Přechod do nastavení ovládání představuje odchycení události stisku
tlačítka Settings v nabídce hry a následné zobrazení nabídky pro nastavení.

\begin{center}
\includegraphics[width=\textwidth]{\picturesfolder/Flow_diagram04_nastaveni.pdf}
\end{center}

\end{document}
